\documentclass[a4paper, 12pt]{amsart}
%\usepackage{amsmath}używa się w klasach, innych niż amsart
\usepackage{amssymb}
\usepackage[utf8]{inputenc}
\usepackage[polish]{babel}
\usepackage[T1]{fontenc}
\author{Filip Kamiński}
\title{Różny rzeczy matematyczny}
\begin{document}
\maketitle
\section{Definiowanie poleceń}
\newcommand{\A}{\mathfrak {A}}
\renewcommand{\P}{\mathcal {P}}
$\P$ ,$\A$
\section{Operatory}

Liczby zespolone są rozszerzeniem liczb rzeczywistych $\mathbb{R}$. Zbiór liczb
zespolonych oznaczamy symbolemi  $\mathbb{C}$ .Każdą liczbę zespoloną z z można
zapisać w postac
\begin{center}
$\ z = a + bi$.
\end{center}

Dla liczby $\ z = a + b$i definiuje się jej:\\
\\
część rzeczywistą re $z = a$ (inne oznaczenia: Re $z$,Re $z$),\\
część urojoną jako im  $z = b$(inne oznaczenia: Im $z$,Im $z$).\\
\\
Dla liczb rzeczywistych oraz zespolonych $\| x\| = | x| = $ $a^{2} + b^{2} $.
Pochodne cząstkowe funkcji f względem zmiennej $x$ oznacza się sym-\\bolami $\frac{\partial f}{\partial x}$ albo $\partial_x  f$

\section{Srodowiska theorem, proof}
\newtheorem{twierdzenie}{Twierdzenie}
\theoremstyle{definicja}
\newtheorem{definicja}{Definicja}
W tekście:
\begin{twierdzenie}[Wielkie twierdzenie Ferma]
Dla liczby naturalnej $n>2$ nie istnieją takie liczby naturalne
dodatnie $x,y,z $ które spełniałyby równanie
\begin{equation}\label{my}
x^{n}+y^{n}=z^{n}.
\end{equation}
\end{twierdzenie}
\begin{proof}
W rzeczywistości dowód twierdzenia Fermata przeprowadzony
przez Wilesa ma dosyć długą historię.
\end{proof}
\begin{definicja}
Liczby naturalne – liczby służące podawaniu liczności
(trzy osoby, zob. liczebnik główny/kardynalny) i ustalania kolejności
(trzecia osoba, zob. liczebnik porządkowy)
\end{definicja}
\begin{definicja}
Trójka pitagorejska – trzy liczby naturalne x, y, z spełniające tzw. równanie Pitagorasa (1).
\end{definicja}
\section{Wzory na kilka wiersze}
Środowisko $multiline$\\
\begin{multline}
|\psi_{1,2}|^2 = 1 + \frac{LC \lambda^2}{2} \pm i \sqrt{-LC \lambda^2 - \left( \frac{LC \lambda^2}{2} \right)^2} \\
= \left( 1 + \frac{LC \lambda^2}{2} \right)^2 - LC \lambda^2 - \left( \frac{LC \lambda^2}{2} \right)^2 = 1.
\end{multline}


Środowisko $align$
\begin{align}
|\psi_{1,2}|^2 &=  1 + \frac{LC \lambda^2}{2} \pm i \sqrt{-LC \lambda^2 - \left( \frac{LC \lambda^2}{2} \right)^2} \tag{3}  \\
&= \left( 1 + \frac{LC \lambda^2}{2} \right)^2 - LC \lambda^2 - \left( \frac{LC \lambda^2}{2} \right)^2 \tag{4} \\
&= 1.
\end{align}
Środowisko $equation, split$
\begin{equation}
\begin{split}
|\psi_{1,2}|^2 &= 1 + \frac{LC \lambda^2}{2} \pm i \sqrt{-LC \lambda^2 - \left( \frac{LC \lambda^2}{2} \right)^2} \\
&= \left( 1 + \frac{LC \lambda^2}{2} \right)^2 - LC \lambda^2 - \left( \frac{LC \lambda^2}{2} \right)^2 \\
&= 1.
\end{split}
\end{equation}
Środowisko $gather$
\begin{gather}
|\psi_{1,2}|^2 = 1 + \frac{LC \lambda^2}{2} \pm i \sqrt{-LC \lambda^2 - \left( \frac{LC \lambda^2}{2} \right)^2} \tag{6} \\
= \left( 1 + \frac{LC \lambda^2}{2} \right)^2 - LC \lambda^2 - \left( \frac{LC \lambda^2}{2} \right)^2 \tag{7} \\
= 1.
\end{gather}



\section{Układy równań}
\begin{equation}
\begin{cases}
x + y = 5, \\
x - 2y = 8.
\end{cases}
\end{equation}
Oraz
\begin{equation}
f(x) =
\begin{cases}
x + 5 & \text{for } x > 0, \\
x^2 + x - 5 & \text{for } x \leq 0.
\end{cases}
\end{equation}

\end{document}

