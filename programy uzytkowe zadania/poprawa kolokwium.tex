
\documentclass[a4paper, 11pt]{amsart}
\usepackage[T1]{fontenc}
\usepackage[polish]{babel}
\usepackage[utf8]{inputenc}
\usepackage{enumitem}
\author[I. Nazwisko]{Imię Nazwisko}
\address{UWM w Olsztynie}
\email{NumerAlbumu@student.uwm.edu.pl}
\title[Kolokwium]{Kolokwium 22.12.2024. Wariant 1}
\begin{document}
\maketitle
\tableofcontents%dodaje spis treści
\section{Tekst}
{\Huge W tym dokumencie są pięc części,odpowiadających rożnym tematom, związanym z \LaTeX}.\texttt{W pierwszej części chodzi o rozmiar trzcionki }. \textsf{W drugiej - o wzorach,potem macierzy,spisy i tabele.}
\section{Wzory}
niech $\Ph=\displaystyle\sum_{k=1}^5 2k^2$.Przypominamy ,że to oznacza,ze
\begin{equation}\label{wzór}
\begin{spit}
\end{document}